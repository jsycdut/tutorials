\documentclass[UTF8]{ctexart}
\usepackage{indentfirst}
\setlength{\parindent}{2em}
\usepackage{hyperref}
\hypersetup{
    colorlinks=true,
    linkcolor=blue,
    filecolor=blue,      
    urlcolor=blue,
    citecolor=cyan,
}

\begin{document}

\section{简介}

\paragraph{}
\indent 此书的编写工作仍在紧锣密鼓的进行中,十分受欢迎来自读者的反馈。\\
\indent 本书的书名意为“给图数据库和Gremlin查询语言的用户的启蒙指导,内容包括Gremlin语言特性,一些小窍门,以及示例查询”。虽然作为一个标题而言这个句子太长,但是很好的概括了本书的主要工作。\\
\indent 我抵制了那种想要在一本参考手册中将Tinkerpop的所有特性一条条的列出来的急切心情,与之相对,我尝试着以一种由易到难的渐进学习的模式来介绍Tinkerpop。为了更好的学习效果,我建议读者在看书的时候一定要先打开Gremlin控制台,加载好示例数据,一边看书,一边做实验。此书没有任何学习门槛,不需要你有Tinkerpop、Gremlin或者相关工具的先验知识,我将从第2章开始所有会被用到的东西。\\
\indent 就目前而言,本书仍在编写中,只要时间允许,内容会愈加丰富,希望本书对它的读者是有用的,比如像我这样正在学习Gremlin 的查询和遍历语言以及相关技术的同伴们。\\
\indent 本书的附加材料,比如本书籍的其他发布格式比如PDF、HTML、ePub和MOBI以及示例代码和数据,可以在 \href{https://github.com/krlawrence/graph}{Github} 上找到,读者将会在"Introducting the book sources, sample programs and data"一节找到所有内容的一个总结。

\subsection{本书的由来}
\indent 大约是一年前,在使用图数据库的时候,尤其是Apache Tinkerpop、JanusGraph和Gremlin,对于其中一些不太明白的地方(并且这些问题也很少被解释清楚)开始编写一些笔记,起初主要是为了自用,然后这些笔记就开始增长,逐渐成为一本书的规模了。在同事们的鼓舞下,我决定将这些笔记发布在一个开源的地方作为在线阅读的书籍,这样任何有兴趣的人都可以读到它。但是这本书的主要目标读者是程序员以及那些使用Gremlin语言的和图打交道的人。大量的例子,示例查询,最优操作的讨论,以及我所学到的内容,全都包含在这本书里面。\\

\indent 我想从心底里对那些鼓励我坚持这段旅程的人们说一声“Thank You”,所有这一切花费了巨大的精力但是也让我乐在其中。

\begin{flushleft}
Kelvin R.Lawrence\\
首批初稿时间:2017年10月5日\\
当前撰稿时间:2018年9月4日\\
\end{flushleft}
\subsection{提供反馈}
\indent 如果本书有任何的错误或者你有任何想法,都可以联系我。要是有建议性的提升内容的话那就更棒了。一个提交反馈的好办法是在\href{https://github.com/krlawrence/graph}{Github}上创建一个issue,本书是第281批预览版。\\
\indent 在此感谢所有创建issue,提交pull request,和花费时间校验本书手稿的人。

\subsection{致谢词}
\indent 在此,感谢我的同事,Graham Wallis,Jason Plurad和AdamHolley,他们改善和提升了本书中一些查询示例的效率。Gremlin毫无疑问是需要团队协作的,我们共同花费了很多时间用于讨论和处理最优的查询方式。\\
\indent 此外,必须感谢\href{https://groups.google.com/forum/#!forum/gremlin-users}{Gremlin Users Google Group}的所有人,是他们抽出宝贵的时间来回复关于Gremlin的很多问题并且提了不少建议。在这里面,必须点名感谢Daniel Kuppitz, Marko Rodriguez以及Stephen Mallette,是他们创建和维护Apache Tinkerpop这个项目。\\
\indent 最后,感谢所有通过e-mail,github-issue, github-pull-request提交反馈的所有人,是你们促使本书不断进化,正如技术本身不断进化一样,非常感谢你们。

\subsection{本书主要内容}
\indent 本书通过以真实的“图数据”为例,介绍了Apache TinkerPop 3 Gremlin 图查询和遍历语言。数据以及其他相关的文件可以从GitHub上下载到。本书用到的图为“air-routes”,该图是一个包含3373个机场在内的43400条飞行路线的抽象模型。该示例能在Gremlin console中通过TinkerGraph加载air-routes.graphml文件来完整运行。如何设置相关的环境在“下载,安装以及启动console控制台”一节详细介绍。该示例已更新且已在Apache TinkerPop 3.3.3版本完成测试。\\

\indent TinkerGraph是一个存在于内存中的图,也就是说相关数据不会持久化到磁盘中。该图属于Apache TinkerPop 3的一部分。本书的意义在于让那些毫无相关技术储备的读者能够上手和快速使用Gremlin console以及ari-routes图。在本书的后续部分,我会介绍使用其他的技术比如JanusGraph,Apache Cassandra,Gremlin Server,以及Elasticsearch来构建可伸缩的和持久化的图存储,并且这些存储仍然可以使用Gremlin查询语言。除了使用Gremlin console外,我还会介绍如何写TinkerPop相关的单独的Java和Groovy程序,当然这里面还混入了几个Ruby程序示例。
\end{document}