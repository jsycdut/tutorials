\documentclass[a4paper]{article}

\usepackage[UTF8]{ctex}
\usepackage{titlesec}
\usepackage{hyperref}
\usepackage{titling}
\usepackage[margin=1in]{geometry}

\hypersetup{
colorlinks=true,
linkcolor=blue,
filecolor=blue,      
urlcolor=blue,
citecolor=cyan,
}

\titleformat{\section}
%{\huge\bfseries}
{\huge}
{\thesection}
{0.5em}
{}[\titlerule]

\titleformat{\subsection}
%{\bfseries\Large}
{\Large}
{$\bullet$}
{0.5em}
{}

\titleformat{\subsubsection}[runin]
%{\bfseries}
{}
{}
{0.5em}
{---}

\renewcommand{\maketitle}{
\begin{center}
{\huge
\theauthor}

Email --- jsycdut@gmail.com

\vspace{.5em}
\end{center}
}

\begin{document}
\title{个人简历}
\author{金世钰}
\date{\today}
\maketitle

\section{学习经历}
\subsection{本科}
2014.09 --- 2018.07,就读于成都理工大学计算机科学与技术专业。
\section{工作经历}
\subsection{蜀天梦图数据科技有限公司}
2018.07 --- 至今,就职于四川蜀天梦图数据科技有限公司,参与公司图数据库产品的研发,包括公司产品的Linux命令行安装程序、交互式终端、图形化管理客户端以及部分核心存储。

\section{工作技能}

\subsection{语言}
\subsubsection{编程语言}
Java

较为熟悉Java,了解常见的Java技术,比如内部类,接口等

\subsubsection{脚本语言}
Bash, JavaScript

能够熟练编写Bash脚本,比如{\href{https://github.com/jsycdut/mac-setup}{OS X平台的软件包安装脚本}}。也能够熟练地使用JavaScript ES6语法,编写JS前端应用。

\subsubsection{标记语言}
{\LaTeX},Markdown

日常使用Markdown做工作笔记,掌握{\LaTeX},可以使用{\LaTeX}编写简单文档(比如这篇简历)

\section{工作流}
vim,arch,git

熟练使用vim编辑器,日常使用Arch Linux(Arch + i3),熟悉git的常用操作。
\end{document}
