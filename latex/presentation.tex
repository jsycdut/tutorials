\documentclass{beamer}
\usepackage[UTF8]{ctex}

\usetheme{Dresden}
\usecolortheme{dolphin}

\title[总结]{图计算组年终总结}
\subtitle{2019回顾与2020展望}
\author{金世钰}
\date{2020年1月8日}
\logo{\LaTeX}

\begin{document}
\maketitle

  % simple group intro
  \begin{frame}
    %\frametitle{substitute this with your title}
      \begin{columns}
         \column{0.98\textwidth}
       \includegraphics[width=\textwidth]{../../photos/bar/cat-selfie-2.jpg}
        \column{0.16\textwidth} OK!让我们组先来一个帅气的自拍!
      \end{columns}
  \end{frame}
  \begin{frame}
    %\frametitle{substitute this with your title}
      \begin{columns}
         \column{0.68\textwidth}
       \includegraphics[width=\textwidth]{../../photos/bar/group.jpg}
       \column{0.25\textwidth} 好吧!这才是真正的我们。
      \end{columns}
  \end{frame}

  % content list
  \begin{frame}
  \frametitle{内容提要}
    \begin{itemize}
      \item{回首2019}
      \item{简要总结}
      \item{展望2020}
    \end{itemize}
  \end{frame}
  
  \begin{frame}
    \frametitle{回首2019}
  \begin{columns}
  \column{0.48\textwidth}
  \includegraphics[height=0.5\textheight]{../../photos/bar/gremlin-lab-coat.png}
  \column{0.52\textwidth}

  \begin{itemize}
  \item 图计算小组的成立,2019.04
  \end{itemize}
  \end{columns}

  \end{frame}
  
  \begin{frame}
    \frametitle{回首2019}
  \begin{columns}
  \column{0.48\textwidth}
  \includegraphics[height=0.5\textheight]{../../photos/bar/commit.png}
  \column{0.52\textwidth}

  \begin{itemize}
  \item 图计算小组的成立,2019.04
  \item 从无到有的图计算,2019.06
  \end{itemize}
  \end{columns}
  \end{frame}


  \begin{frame}
    \frametitle{回首2019}
  \begin{columns}
  \column{0.48\textwidth}
  \includegraphics[width=\textwidth]{../../photos/bar/neo4j-ha.png}
  \column{0.52\textwidth}

  \begin{itemize}
  \item 图计算小组的成立,2019.04
  \item 从无到有的图计算,2019.06
  \item 单机图计算的增强,2019.11
  \end{itemize}
  \end{columns}

  \end{frame}


  \begin{frame}
    \frametitle{回首2019}
  \begin{columns}
  \column{0.48\textwidth}
  \includegraphics[height=0.7\textheight]{../../photos/gdm/tp-dis.png}
  \column{0.52\textwidth}

  \begin{itemize}
  \item 图计算小组的成立,2019.04
  \item 从无到有的图计算,2019.06
  \item 单机图计算的增强,2019.11
  \item 从单机走向分布式,2019.12
  \end{itemize}
  \end{columns}

  \end{frame}

  \begin{frame}
    \frametitle{回首2019}
  \begin{columns}
  \column{0.48\textwidth}
  \includegraphics[height=0.65\textheight]{../../photos/bar/raft.png}
  \column{0.52\textwidth}

  \begin{itemize}
  \item 图计算小组的成立,2019.04
  \item 从无到有的图计算,2019.06
  \item 单机图计算的增强,2019.11
  \item 从单机走向分布式,2019.12
  \end{itemize}
  \end{columns}

  \end{frame}

  \begin{frame}
    \frametitle{简要总结}
    在过去的一年里,我们有做的比较好的地方,也存在很多不足

    \begin{columns}
      \column{0.5\textwidth}
        \includegraphics[scale=0.35]{../../photos/bar/smile.png}
        \begin{itemize}
          \item 严格把控计划进度
          \item 规范的git提交记录
          \item 设计模式和语言特性的大量使用
        \end{itemize}
      \column{0.5\textwidth}
        \includegraphics[scale=0.35]{../../photos/bar/cry.png}
        \begin{itemize}
          \item 缺少全局性的思考
          \item 数据结构使用不合理
          \item 日志输出的重视程度不够
        \end{itemize}
    \end{columns}
  \end{frame}

  % 2020
  \begin{frame}
  \frametitle{展望2020}
    \begin{columns}
      \column{0.48\textwidth}
        \includegraphics[width=\textwidth]{../../photos/bar/big-data.png}
      \column{0.52\textwidth}
      \begin{itemize}
        \item 支持10亿级别的分布式图数据计算
      \end{itemize}
    \end{columns}
  \end{frame}

  \begin{frame}
  \frametitle{展望2020}
    \begin{columns}
      \column{0.48\textwidth}
        \includegraphics[width=\textwidth]{../../photos/bar/gremlin-apache.png}
      \column{0.52\textwidth}
      \begin{itemize}
        \item 支持10亿级别的分布式图数据计算
        \item 扩充和优化Tinkerpop的图计算算法
      \end{itemize}
    \end{columns}
  \end{frame}

  \begin{frame}
  \frametitle{展望2020}
    \begin{columns}
      \column{0.48\textwidth}
        \includegraphics[width=\textwidth]{../../photos/bar/gremlin-apache.png}
      \column{0.52\textwidth}
      \begin{itemize}
        \item 支持10亿级别的分布式图数据计算
        \item 扩充和优化Tinkerpop的图计算算法
      \end{itemize}
    \end{columns}
  \end{frame}
\end{document}
