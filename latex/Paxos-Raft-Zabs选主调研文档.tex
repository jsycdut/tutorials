\documentclass{article}
\usepackage{multicol}
\usepackage[UTF8]{ctex}
\usepackage{xcolor}
\usepackage{abstract}
\usepackage{tikz}
\usepackage{hyperref}
\usetikzlibrary{graphs}
\usetikzlibrary{arrows, patterns, positioning}

\title{Paxos-Raft-Zab算法调研文档}
\author{金世钰,\textbf{四川蜀天梦图数据科技有限公司}}

% 双栏设置
\setlength{\columnsep}{1cm}

% 超链接设置
\hypersetup{
    colorlinks=true,
    linkcolor=blue,
    filecolor=magenta,      
    urlcolor=cyan,
}

\begin{document}
\maketitle

\begin{abstract}
本文主要介绍了Paxos、Raft和Zab三种目前业界主流采用的共识算法,为分布式图计算主节点选举提供备选算法方案。
\end{abstract}
\begin{multicols}{2}

\section{前言}
目前,图计算小组开发的分布式图计算模型基于原有Tinkerpop所采用的BSP计算模型,已经能够成功使用多个节点,共同执行Pagerank算法并得出正确结果。其计算模型大致如下图所示。

\begin{center}
\begin{tikzpicture}[scale=0.5,
      blue circle/.style={circle, draw=blue!50, fill=blue!20, thick, minimum size=6mm},
      transition/.style={rectangle, draw=black!75,fill=black!20},
      red circle/.style={blue circle, draw=red!50, fill=red!20},
      green circle/.style={blue circle, draw=green!20, fill=green!50}
    ]
	\draw[help lines] (-5, -5) grid(5, 5);
        \draw[thick, red, -latex] (-5, 0) -- (5, 0);
        \draw[thick, red, -latex] (0, -5) -- (0, 5);
	    \draw[color=green!20] (-5, -4) rectangle (5, -2);
	    \node[blue circle] (client) at (0, 4) {客户端};
	    \node[red circle] (leader) at (0, 1) {主节点}; % edge[latex-, bend left=10, thick] node[sloped, anchor=center, above, text width=1.2cm] {\tiny{提交计算任务}} (client) edge[-latex, bend right=10, thick] node[sloped, anchor=center, above, text width=1.2cm] {\tiny{返回计算结果}} (client);
	    \node[green circle] (f1) at (-4, -3) {从节点};
	    \node[green circle] (f2) at (0, -3) {从节点}; %edge[latex-, bend left=10, thick] node[sloped, anchor=center, above, text width=1.6cm] {\tiny{接收超步计算消息}} (leader) edge[-latex, bend right=10, thick] node[sloped, anchor=center, below, text width=1.6cm] {\tiny{返回超步计算结果}} (leader);
	    \node[green circle] (f3) at (4, -3) {从节点};
    \end{tikzpicture}
\end{center}


该模型涉及到一个协调各节点的leader节点,但是在该模型中,leader节点存在单点问题,当leader节点下线时(节点故障、重启或者被网络隔离开),集群便陷入“群龙无首”的困境,此时分布式计算任务无法派发,整个计算流程就卡死在第一步。所以需要一个算法,使得集群中leader节点下线时,能够自动选举出新的leader节点,从而解决leader节点的单点问题。

本次调研主要是为分布式图计算主节点选举提供参考方案,调研了当前分布式系统用到的主流算法,包括Paxos、Raft和Zab三种算法。这三种算法都是分布式共识算法,本身内容不仅仅局限于集群中主节点选举这一部分,但是根据我们分布式图计算系统的需要,我们暂时只关注于算法中的主节点选举部分,或者只将算法用于主节点选举。 

\section{算法介绍}

\subsection{Paxos}
Paxos算法由2013年图灵奖获得者\emph{Leslie Lamport}在1990率先提出,该算法主要是用于解决分布式系统的一致性问题,也就是分布式中所有成员对某个提议如何达成共识的问题。

Paxos算法在1990年时并没有得到发表,起因是该算法以一种类似于拜占庭将军的形式对算法进行描述,当时的编辑拒绝对此文进行发表,直到8年之后的1998年,有一个团队需要在分布式系统中达成共识,此时Lamport将算法给了他们并且得到了实施,尔后Lamport再次发表了该算法,也就是著名的《The Part-Time Parliament》,但是对读者来说,该故事性的算法描述却仍然难以理解,直到2001年,Lamport再次修改该算法的表述,发表了论文《Paxos Made Simple》,清晰的解释了该算法。随后,Google公司的Mike Burrows在2006年发表了论文《The Chubby lock service for loosely-coupled distributed systems》,其中就使用了Paxos算法作为其一致性算法,此后Paxos算法受到热烈欢迎。

本文主要参考了论文\href{https://lamport.azurewebsites.net/pubs/paxos-simple.pdf}{《Paxos Made Simple》}

Paxos作为分布式系统中的共识性算法,其主要解决的便是分布式系统中的成员就某事如何达成共识,在分布式图计算中,这个共识就是:谁是主节点?

接下来,主要介绍Paxos算法如何达成一致。

Paxos算法有两个确保共识的要求(论文中表述为三个,但是其第二和第三点其实是同一点),分别是
	\begin{enumerate}
		\item 被提议的值可以有多个,但是最终只会有一个被选择
		\item 只有一个值会被最终选择,并且被选中的值只有被真正选中之后才会被告知给其他所有的成员
	\end{enumerate}

在整个算法中,所有成员可以被归类为三种角色,分别是
	\begin{itemize}
		\item proposers,负责提议某个值
		\item acceptors,负责检验proposers提出的值
		\item learners,负责告知所有成员最后选定的值
	\end{itemize}

成员之间可以通过互发消息进行通信,整个消息传递使用异步的消息,并且是非拜占庭模型,也就是说
	\begin{itemize}
		\item 成员可以停机、重启,成员需要能够保存住某些信息;
		\item 消息传递需要经历一定的时间段,消息可以重复,可以丢失,但是消息一定是完整的。
	\end{itemize}

	在整个系统中,有以下几点要求

	\begin{enumerate}
		\item accptors必须接受第一个被提议的值


\subsection{Raft}

Raft一般指由斯坦福大学的Diego Ongaro和 John Ousterhout在2014年发表的博士论文《In Search of an Understandable Consensus Algorithm》中提出的分布式系统共识算法。

该论文主要包括主节点选举和日志复制两部分,其中和我们分布式图计算有关的是主节点选举。



\subsection{Zab}





\end{multicols}

\end{document}
